% ============================================================================
% PAPER TABLES — Include in Main Document
% ============================================================================
% 
% These three tables form the complete quantitative backbone of the paper.
% No redundancy. All justified. Placed at optimal locations per outline.
%
% Usage in main paper:
%   \section{Baseline Characterization: The Unoptimized State}
%   \subsection{Experimental Setup}
%   \begin{table}[h]
\centering
\caption{Benchmark Configuration: Hardware, Software, Model, and Evaluation Setup}
\label{tab:benchmark_configuration}
\begin{tabular}{ll}
\toprule
\textbf{Category} & \textbf{Specification} \\
\midrule
\multicolumn{2}{l}{\textbf{Hardware}} \\
GPU & NVIDIA RTX A6000 (48 GB VRAM) \\
GPU Memory Bandwidth & 768 GB/s \\
GPU Peak Throughput & 312 TFLOPS (bfloat16) \\
Clock Locking & Enabled (P0 max clocks) \\
\midrule
\multicolumn{2}{l}{\textbf{Software Environment}} \\
Python Version & 3.11.14 \\
PyTorch Version & 2.9.1+cu128 \\
CUDA Version & 12.8 \\
Precision & bfloat16 (mixed precision) \\
\midrule
\multicolumn{2}{l}{\textbf{Model Configuration}} \\
Model & GPT-2 Medium (Pretrained) \\
Total Parameters & 378.96M \\
Number of Layers & 24 \\
Number of Attention Heads & 16 \\
Embedding Dimension & 1024 \\
Block Size (Context Length) & 1024 \\
Vocabulary Size & 50,257 \\
Dropout & 0.0 (evaluation) \\
\midrule
\multicolumn{2}{l}{\textbf{Benchmark Configuration}} \\
Batch Size & 1 (single sample, autoregressive) \\
Sequence Lengths (T) & 32, 64, 96, 128, 160, 192, 224, 256, 320, 384, \\
 & 448, 512, 640, 768, 896, 1024 \\
Warmup Iterations & 5 (per benchmark, per version) \\
Measurement Repetitions & 10 (per T value, per version) \\
\midrule
\multicolumn{2}{l}{\textbf{Dataset and Evaluation}} \\
Dataset & WikiText-2 (test split) \\
Total Tokens Available & 287,644 \\
Tokens per Evaluation & 4,096 (autoregressive, chunked) \\
Evaluation Mode & Autoregressive (one token at a time) \\
Chunking Strategy & Independent ≤1024-token chunks (resets KV cache) \\
\midrule
\multicolumn{2}{l}{\textbf{Measurement Details}} \\
Timing Method & torch.cuda.Event (GPU wall-clock) \\
Memory Measurement & torch.cuda.memory\_reserved() \\
Metrics Synchronized & Yes (torch.cuda.synchronize()) \\
Deduplication Strategy & Keep latest run per (benchmark, version, T) \\
\bottomrule
\end{tabular}
\\\vspace{0.3cm}
{\small \textit{This configuration ensures reproducibility and fair comparison across all optimization variants. Clock locking and warmup iterations eliminate timing variance. The chunked evaluation strategy avoids sliding window position embedding artifacts discovered during development.}
}
\end{table}

%
%   \subsection{Roofline Analysis}
%   \begin{table}[h]
\centering
\caption{Roofline Analysis: Performance Characterization Against Hardware Limits}
\label{tab:roofline_metrics}
\begin{tabular}{lcccccc}
\toprule
\textbf{Scenario} & \textbf{DRAM (GB)} & \textbf{FLOPs (M)} & \textbf{AI} & \textbf{Achieved (GF/s)} & \textbf{Theoretical (GF/s)} & \textbf{Status} \\
\midrule
v0 Prefill & 1.48 & 2320 & 1.57 & 377.0 & 1206 & Mem-bound \\
v1 Prefill & 1.48 & 2320 & 1.56 & 375.0 & 1197 & Mem-bound \\
v1 Decode & 0.88 & 448 & 0.51 & 146.3 & 392 & Mem-bound \\
\bottomrule
\end{tabular}
\\\vspace{0.3cm}
{\small \textit{AI:} Arithmetic Intensity (FLOPs/byte). \textit{Achieved:} Measured throughput. \textit{Theoretical:} Maximum at this AI (min of memory bandwidth and compute rooflines). All scenarios are memory-bound (below compute limit of 312 TFLOPS). Ridge point (AI crossing) is at 406 FLOPs/byte. The low achieved throughput relative to theoretical reveals memory subsystem inefficiency—a key optimization target.}
}
\end{table}

%
%   \section{Evaluation: Performance, Quality, and Trade-offs}
%   \subsection{Evaluation Protocol}
%   \begin{table}[h]
\centering
\caption{Core Results Summary (Table B): KV-Cache Optimization Performance at T=1024}
\label{tab:core_results}
\begin{tabular}{l|cccccc}
\toprule
Version & Latency (ms) & Per-Token (ms) & VRAM (MB) & PPL & Max Batch & Speedup \\
\midrule
v0 & 7362.5 & 7.19 & 799.2 & 22.60 & 7586 & 1.00$\times$ \\
\textbf{v1} & \textbf{3347.9} & \textbf{3.27} & \textbf{920.8} & \textbf{22.54} & 491 & \textbf{2.20$\times$} \\
v2 & 6754.1 & 6.60 & 845.4 & 22.58 & 962 & 1.09$\times$ \\
v3 & 3099.4 & 3.03 & 845.3 & \textbf{988.76} & 970 & 2.38$\times$ \\
v4 & 5273.4 & 5.15 & 807.6 & 997.09 & \textbf{1867} & 1.40$\times$ \\
\bottomrule
\end{tabular}
\\\vspace{0.2cm}
{\footnotesize \textit{GPU:} NVIDIA RTX A6000 (48GB). \textit{Model:} GPT-2 Medium (378.96M params, 24 layers, block\_size=1024). \textit{Eval:} Autoregressive on WikiText-2 test split (287,644 tokens), 4096 token window. \textit{Speedup:} vs v0 baseline (no cache).}
\end{table}
%
% ============================================================================

\documentclass[11pt]{article}
\usepackage{booktabs}
\usepackage{geometry}
\geometry{margin=1in}

\title{Paper Tables — LaTeX Source}
\author{}
\date{}

\begin{document}

\section{Table 1: Benchmark Configuration (§4.1)}

\begin{table}[h]
\centering
\caption{Benchmark Configuration: Hardware, Software, Model, and Evaluation Setup}
\label{tab:benchmark_configuration}
\begin{tabular}{ll}
\toprule
\textbf{Category} & \textbf{Specification} \\
\midrule
\multicolumn{2}{l}{\textbf{Hardware}} \\
GPU & NVIDIA RTX A6000 (48 GB VRAM) \\
GPU Memory Bandwidth & 768 GB/s \\
GPU Peak Throughput & 312 TFLOPS (bfloat16) \\
Clock Locking & Enabled (P0 max clocks) \\
\midrule
\multicolumn{2}{l}{\textbf{Software Environment}} \\
Python Version & 3.11.14 \\
PyTorch Version & 2.9.1+cu128 \\
CUDA Version & 12.8 \\
Precision & bfloat16 (mixed precision) \\
\midrule
\multicolumn{2}{l}{\textbf{Model Configuration}} \\
Model & GPT-2 Medium (Pretrained) \\
Total Parameters & 378.96M \\
Number of Layers & 24 \\
Number of Attention Heads & 16 \\
Embedding Dimension & 1024 \\
Block Size (Context Length) & 1024 \\
Vocabulary Size & 50,257 \\
Dropout & 0.0 (evaluation) \\
\midrule
\multicolumn{2}{l}{\textbf{Benchmark Configuration}} \\
Batch Size & 1 (single sample, autoregressive) \\
Sequence Lengths (T) & 32, 64, 96, 128, 160, 192, 224, 256, 320, 384, \\
 & 448, 512, 640, 768, 896, 1024 \\
Warmup Iterations & 5 (per benchmark, per version) \\
Measurement Repetitions & 10 (per T value, per version) \\
\midrule
\multicolumn{2}{l}{\textbf{Dataset and Evaluation}} \\
Dataset & WikiText-2 (test split) \\
Total Tokens Available & 287,644 \\
Tokens per Evaluation & 4,096 (autoregressive, chunked) \\
Evaluation Mode & Autoregressive (one token at a time) \\
Chunking Strategy & Independent ≤1024-token chunks (resets KV cache) \\
\midrule
\multicolumn{2}{l}{\textbf{Measurement Details}} \\
Timing Method & torch.cuda.Event (GPU wall-clock) \\
Memory Measurement & torch.cuda.memory\_reserved() \\
Metrics Synchronized & Yes (torch.cuda.synchronize()) \\
Deduplication Strategy & Keep latest run per (benchmark, version, T) \\
\bottomrule
\end{tabular}
\\\vspace{0.3cm}
{\small \textit{This configuration ensures reproducibility and fair comparison across all optimization variants. Clock locking and warmup iterations eliminate timing variance. The chunked evaluation strategy avoids sliding window position embedding artifacts discovered during development.}
}
\end{table}


\pagebreak

\section{Table 2: Roofline Metrics (§4.5)}

\begin{table}[h]
\centering
\caption{Roofline Analysis: Performance Characterization Against Hardware Limits}
\label{tab:roofline_metrics}
\begin{tabular}{lcccccc}
\toprule
\textbf{Scenario} & \textbf{DRAM (GB)} & \textbf{FLOPs (M)} & \textbf{AI} & \textbf{Achieved (GF/s)} & \textbf{Theoretical (GF/s)} & \textbf{Status} \\
\midrule
v0 Prefill & 1.48 & 2320 & 1.57 & 377.0 & 1206 & Mem-bound \\
v1 Prefill & 1.48 & 2320 & 1.56 & 375.0 & 1197 & Mem-bound \\
v1 Decode & 0.88 & 448 & 0.51 & 146.3 & 392 & Mem-bound \\
\bottomrule
\end{tabular}
\\\vspace{0.3cm}
{\small \textit{AI:} Arithmetic Intensity (FLOPs/byte). \textit{Achieved:} Measured throughput. \textit{Theoretical:} Maximum at this AI (min of memory bandwidth and compute rooflines). All scenarios are memory-bound (below compute limit of 312 TFLOPS). Ridge point (AI crossing) is at 406 FLOPs/byte. The low achieved throughput relative to theoretical reveals memory subsystem inefficiency—a key optimization target.}
}
\end{table}


\pagebreak

\section{Table 3: Core Results Summary (§7.0)}

\begin{table}[h]
\centering
\caption{Core Results Summary (Table B): KV-Cache Optimization Performance at T=1024}
\label{tab:core_results}
\begin{tabular}{l|cccccc}
\toprule
Version & Latency (ms) & Per-Token (ms) & VRAM (MB) & PPL & Max Batch & Speedup \\
\midrule
v0 & 7362.5 & 7.19 & 799.2 & 22.60 & 7586 & 1.00$\times$ \\
\textbf{v1} & \textbf{3347.9} & \textbf{3.27} & \textbf{920.8} & \textbf{22.54} & 491 & \textbf{2.20$\times$} \\
v2 & 6754.1 & 6.60 & 845.4 & 22.58 & 962 & 1.09$\times$ \\
v3 & 3099.4 & 3.03 & 845.3 & \textbf{988.76} & 970 & 2.38$\times$ \\
v4 & 5273.4 & 5.15 & 807.6 & 997.09 & \textbf{1867} & 1.40$\times$ \\
\bottomrule
\end{tabular}
\\\vspace{0.2cm}
{\footnotesize \textit{GPU:} NVIDIA RTX A6000 (48GB). \textit{Model:} GPT-2 Medium (378.96M params, 24 layers, block\_size=1024). \textit{Eval:} Autoregressive on WikiText-2 test split (287,644 tokens), 4096 token window. \textit{Speedup:} vs v0 baseline (no cache).}
\end{table}

\end{document}
